\section{Introduction}

%Place notes here for the time being
\iffalse
{Talk about what the three different categories are, and how they can be differentiated from each other.  (this should be about 2 paragraphs).  
Then talk about what commonalities each category has with the others(this can be again 1-2 paragraphs) }
\fi


{In the Natural Language Processing community there is currently an open frontier of research tasks focused on ways in which texts can be transformed from their original versions into some form of distilled output.  The general goal of these tasks is to reduce a single text or several texts in such a way that important information is preserved but some aspect of complexity is reduced  Over the course of this paper, I intend to give an overview of three of these tasks:text compression, text simplification, and multi-document text paraphrasing / sentence fusion.  While exploring these tasks, I also intend to focus on the use of discourse analysis,when applicable, as it represents a promising means of identify important information in these compression related tasks.}


{Let us then begin with some basic definitions of what each task entails.  Text compression was initially defined as a sentence level task with the goal of producing a sentence summary which preserves the most important information and remains grammatical \citet{Jing2000}.  This idea was expanded by \citet{Clarke&Lapata2010} by using discourse constraints to have document level information to inform sentnece compression.  }


 {as a reduced text is one in which the total number of words is reduced, while still preserving the important information and retaining grammaticality.  In the task of text simplification, a reduced text would be semantically and/or syntactically less complex, and may also be reduced in length (though this is not a necessary condition).  In the task of paraphrasing, a collection of texts is searched for equivalent sentences which represent important information. From thes sentences an abstractive summary can be generated by fusing pieces of these sentences together.
Over the course of this paper I will highlight the commonalities and differences among these tasks, and focus special attention on how discourse information has been used in each task.}


{single text or multiple texts can be compressed, simplified or fused together.  This field of research can be seen, in a general sense, as ways in which a source text (or collection of texts) can be transformed into a less complex, more managable output text designed to meet some goal.  For the sake of consistience I will refer to these tasks as  Because of the diverse end goals a researcher may be seeking by reducing a text's complexity, the methods, data representations,} 





 :

\t{Because of the diverse end goals a researcher may be aiming for, each of the three tasks (compression, simplification, paraphrasing) approach the goal of transforming the source text in a different manner. 
}







\subsection{Goals}
{Here talk about the Goals of each category, and how that has come about over the course of research on the subject}

 {The three tasks all share a common goal of reducing a text from it's original form into a more manageable and useful format suited to a specific need.   In the case of text compression, texts are reduced in order to improve reading times,\citet{bla} or to adapt text to smaller devices\citet{bla}. For Text simplification, the goal may be to reduce a text's complexity for children\citet{bla}.or those with reading imparements \citet{bla}.  It has also been recently used for improving medical document information retrieval \citet{bla}. In the case of mulit-document text paraphrasing,  systems are  designed to reduce text from multiple sources in order to create an abstractive summary.\citet{bla}.}

\subsection{Approaches}
{Talk about the way that the problems are modeled for each category of text reduction. Then talk about some of the algorithms (and possibly machine learning paradigms) used to solve the tasks of compression, simplification, and paraphrasing.  When there is overlap point it out, and also make note of when there are drastically diverging. }


\subsection{Data}

{Highlight the kind of data used to train and/or model the problem for each text reduction task. Is the data directly used to train a system, or is it simply used as a frame of reference for un-supervised learning.  What kind of data is used for validation and evaluating the systems? }

\subsection{Use of Discourse Information}
{Go into depth about what systems make use of Discourse level informaiton, either directly in processing of the text, or perhaps in a more limited aspect in the evaluation of the output.  Also mention versions of text reduction that do not make use of any Discourse information.  Are they any better?  Is discourse information at this point not terribly helpful to solving the task?}

\subsection{Natbib citations}
Within a text, you can say that \citet{lin2001} found out something. Or you can just state the thing, and then put the author in parentheses \citep[see][]{szpektor2004}.